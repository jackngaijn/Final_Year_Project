\documentclass[12pt,a4paper]{article}
\usepackage[utf8]{inputenc} % For UTF-8 encoding
\usepackage{amsmath, amssymb} % For mathematical symbols
\usepackage{graphicx} % For including images
\usepackage{hyperref} % For clickable links
\usepackage{geometry} % For page margins
\usepackage{booktabs} % For nice tables
\usepackage{lipsum} % For dummy text

% Page layout
\geometry{margin=1in}

% Hyperlink setup
\hypersetup{
    colorlinks=true,
    linkcolor=blue,
    filecolor=magenta,
    urlcolor=cyan,
    pdftitle={LaTeX Template},
    pdfpagemode=FullScreen,
}

% Title and author
\title{Recurrent Neural Networks and its applications}
\author{Ngai Ho Wang}
\date{\today} % Automatically insert today's date

\begin{document}

% Title page
\maketitle

\tableofcontents % Generate a table of contents
\newpage

% Section 1
\section{Introduction}
This is the introduction section. You can write about the purpose of the document, its structure, or any background information. For example:

\lipsum[1] % Dummy text (remove this and add your content)

\section{Mathematical Examples}
Below are examples of how to include mathematical formulas and equations.

\subsection{Inline Formula}
An example of an inline formula: \( E = mc^2 \).

\subsection{Displayed Equations}
Here is a displayed equation:
\[
\int_{a}^{b} x^2 \, dx = \frac{b^3}{3} - \frac{a^3}{3}
\]

\subsection{Equation Numbering}
You can also number your equations:
\begin{equation}
    F = ma
\end{equation}

\section{Inserting Images}
To insert an image, use the `graphicx` package. For example:

\begin{figure}[h!]
    \centering
    \includegraphics[width=0.5\textwidth]{example-image} % Replace "example-image" with your image file
    \caption{An example image.}
    \label{fig:example}
\end{figure}

\section{Tables}
You can create tables using the `tabular` environment or `booktabs` for professional-quality tables. For example:

\begin{table}[h!]
\centering
\caption{Example Table}
\begin{tabular}{@{}llr@{}}
\toprule
\textbf{Item} & \textbf{Description} & \textbf{Quantity} \\ \midrule
Apples        & Fresh red apples     & 10                \\
Oranges       & Juicy oranges        & 5                 \\
Bananas       & Ripe bananas         & 7                 \\ \bottomrule
\end{tabular}
\label{tab:example}
\end{table}

\section{Hyperlinks}
To add a hyperlink, use the `hyperref` package. For example:
\href{https://www.latex-project.org/}{Visit the LaTeX project website}.

\section{Conclusions}
This is the conclusion section. Summarize your findings or leave final remarks.

\newpage
\appendix
\section{Appendix}
This is the appendix section, where you can include supplementary materials.

\end{document}